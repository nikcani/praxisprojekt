\newpage


\section*{Kurzzusammenfassung}
Radiosonden sind Instrumente zur Messung meteorologischer bzw.\ aerologischer Daten aus der Erdatmosphäre.
Für die Verwaltung und Auswertung von mehreren Bodenstationen und deren Radiosondenaufstiegen wird eine webbasierte Software entwickelt.
Ziel dieser Arbeit ist die Konzeption und Implementierung eines ersten lauffähigen, sowie produktiv nutzbaren Prototypen der Software.
Außerdem sollen zwei Hostingvarianten, Cloud based und On-Premises, umgesetzt werden.

Im Rahmen dieser praktischen Arbeit wird evaluiert, welche Vor- und Nachteile der Einsatz eines Admin-Panel Frameworks, am konkreten Beispiel von Laravel Nova, bietet.
Zusätzlich wird ein Vergleich zu ähnlichen Technologien, im methodischen Rahmen einer kurzen Nutzwertanalyse, gezogen.

Laravel Nova überzeugt durch stark gesteigerte Produktivität während der Implementierung.
In puncto Anpassbarkeit müssen dafür jedoch einige Einschränkungen in Kauf genommen werden.
Der Verglich mit ähnlichen Frameworks zeigt, dass es möglicherweise besser geeignete Kandidaten gibt.
