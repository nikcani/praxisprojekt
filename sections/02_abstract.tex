\newpage


\section*{Abstract}
Radiosonden sind Instrumente zur Messung meteorologischer bzw.\ aerologischer Daten in der Erdatmosphäre.
Für die Verwaltung und Auswertung von mehreren Bodenstationen sowie deren Radiosondenaufstiegen sollte eine webbasierte Software entwickelt werden.
Ziel dieser Projektarbeit ist die Konzeption und Implementierung eines ersten lauffähigen und produktiv nutzbaren Prototypen der Software.
Außerdem wurden zwei Hostingvarianten, Cloud based und On-Premises, umgesetzt.

Im Rahmen dieser praktischen Arbeit wird evaluiert, welche Vor- und Nachteile der Einsatz eines Admin-Panel Frameworks, am konkreten Beispiel von Laravel Nova, bietet.
Zusätzlich wird ein Vergleich zu ähnlichen Technologien, im methodischen Rahmen einer kurzen Nutzwertanalyse, erstellt.

Im Ergebnis überzeugt Laravel Nova durch stark gesteigerte Produktivität während der Implementierung.
In puncto Anpassbarkeit müssen dafür jedoch einige Einschränkungen in Kauf genommen werden.
Der Vergleich mit ähnlichen Frameworks zeigt, dass es möglicherweise besser geeignete Lösungen gibt.
