\section{Einleitung}

\subsection{Motivation und Vorgehensweise}
Das Praxisprojekt findet in enger Kooperation mit der Digitalagentur \enquote{Kiwis \& Brownies - Christian Kiewaldt \& Benjamin Braun GbR}\cite{kiwis}, sowie dem Auftraggeber des Projektes, der \enquote{GRAW Radiosondes GmbH \& Co. KG}\cite{graw}, statt.

GRAW Radiosondes stellt meteorologische Messinstrumente her, vor allem Radiosonden, die getragen von einem Wetterballon, bis in Höhen der Stratosphäre aufsteigen und während des Fluges verschiedene Messdaten sammeln.
Das Ziel dieser Arbeit, seitens GRAW, ist die Erprobung und Entwicklung einer Managementsoftware, mit deren Hilfe Stationen, deren Flüge und wiederum deren Messdaten, archiviert, verwaltet und ausgewertet werden können.

Bisherige individuelle Projekte des Auftragnehmers Kiwis \& Brownies zeigten, laut Gesprächen mit Management und Entwicklern, einen anfänglich großen und repetitiven Aufwand bei der Erstellung individueller Verwaltungssoftware auf.
Fast alle Projekte benötigen einen Login, die Verwaltung von Entitäten in einem sogenannten \enquote{CRUD}\cite{crud} Interface, verschiedene Auswertungen, sowie spezielle Abläufe für eine oder mehrere Entitäten.

Diese Praxisprojektarbeit soll einen ersten lauffähigen, sowie produktiv nutzbaren Prototypen der Software entwerfen und implementieren.
Um den Umfang der Software möglichst effizient umzusetzen, soll der Einsatz eines unterstützenden Frameworks erprobt werden, welches das Grundgerüst stellt und wiederkehrende Funktionen schneller konfigurierbar macht und insgesamt auch für eine einheitliche Software sorgt.
Dabei ist zu prüfen, an welchen Stellen das zu erprobende Framework \enquote{Laravel Nova}\cite{laravel-nova} bei der Erstellung der Software hilfreich ist und welche Einschränkungen sich durch die Verwendung ergeben.
\newpage

\subsection{Projektfragen}
Aus Entwicklungssicht ergibt sich die konkrete Frage:
\enquote{Welche Vor- und Nachteile bietet der Einsatz eines Admin Panels am konkreten Beispiel von Laravel Nova?}

Auf Basis dieser Kernfrage lassen sich weitere Fragen für das Projekt ableiten:
\begin{itemize}
    \item Lässt sich die Sounding Console grundsätzlich mit Nova umsetzen, oder gibt es zu starke Einschränkungen bei der Umsetzung der Anforderungen?
    \item Welche Anforderungen der Sounding Console lassen sich durch Nova besonders gut entwickeln?
    \item Bei welchen Anforderungen entstehen Nachteile durch die Verwendung von Nova?
    \item Lässt sich abschätzen, ob zukünftige Erweiterungen gut mit Nova umsetzbar sind?
    \item Können am vorliegenden Beispiel Muster erkannt werden, welche Anforderungsbereiche betroffen sind, um auf Basis dessen zukünftige Projektentscheidungen zu treffen?
    \item Wie flexibel ist der Technologie Stack und wie sehr wird man durch Nova eingeschränkt?
\end{itemize}
