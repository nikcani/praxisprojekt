\section{Konzeption}
Erfahrungswerte zeigen, dass die kritischsten Technologieentscheidungen im Bereich der Webtechnologien das Backend betrifft, vorausgesetzt, es gibt keine speziellen Anforderungen des Frontends die berücksichtigt werden müssen.
Daher starten die Entscheidungen im Backend und anschließend wird das Frontend festgelegt.
Auf Basis des Backends und des Frontends, sowie weiterer Projektanforderungen, wird ein passendes Hosting ausgewählt.

\subsection{Backend}

\paragraph{PHP}
still the web standard, dev experience

\paragraph{Laravel}
\enquote{batteries included}, lots of packages like nova, large community, well documented, dev experience, current industry standard for modern php apps\cite{laravel-nova-docs}

\paragraph{Datenbank}
egal, hauptsache relational, standard und support durch laravel

\subsubsection{Frontend}

\paragraph{Chart.js}
gute doku, umfangreich aber nicht zu umfangreich
Vergleiche d3 und google charts

\paragraph{Polling}
Polling anstatt sockets, weil keine wirkliche Echtzeit notwendig, alle 5 sekunden genügt und erzeugt vermutlich weniger last auf dem server als

\subsubsection{Hosting}

\paragraph{vServer}

\subparagraph{Laravel Forge}
Durch den Einsatz von Laravel bietet sich die Verwendung von \enquote{Laravel Forge}\cite{laravel-forge}, einem First Party Server Management Tool, an.
Im aktuellen Projekt ergaben sich durch die Nutzung von Forge vor allem folgende Vorteile:
\begin{itemize}
    \item ein moderner und robuster Hosting Stack wird automatisiert verwaltet
    \item Push To Deploy ermöglicht schnelle Updates in Test- und Produktivsysteme
    \item automatisiert eingebundene TLS Zertifikate mittels LetsEncrypt
    \item Task Scheduling: Übersichtliche und schnelle Konfiguration durch Entwickler möglich
    \item Log File Access: gesammelter Zugriff auf alle relevanten Logfiles inklusive \enquote{Log Tailing}
\end{itemize}

Zukünftig können folgende Features von Vorteil sein:
\begin{itemize}
    \item Automatisierte Datenbankbackups
    \item Laravel Octane Support - neues Paradigma durch Stateful Server und dadurch schnellere Application Performance\cite{laravel-octane}
\end{itemize}

Als Hostinganbieter unterstützt Forge

\subparagraph{Hetzner Cloud}
- scaling
- managed backups möglich
- kein container overhead, volle Leistung der gebuchten Ressourcen
- gut durch den entwickler selbst betreubar

\paragraph{Containerized}
based on docker and kubernetes, written in terraform
- better scaling
- IaC (Infrastructure as Code)
