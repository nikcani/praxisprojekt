\section{Konzeption}
Erfahrungswerte zeigen, dass die kritischsten Technologieentscheidungen im Bereich der Webtechnologien das Backend betrifft, vorausgesetzt, es gibt keine speziellen Anforderungen des Frontends die berücksichtigt werden müssen.
Daher starten die Entscheidungen im Backend und anschließend wird das Frontend festgelegt.
Auf Basis des Backends und des Frontends, sowie weiterer Projektanforderungen, wird ein passendes Hosting ausgewählt.

\subsection{Backend}

\paragraph{PHP}
still the web standard, dev experience

\paragraph{Laravel}
\enquote{batteries included}, lots of packages like nova, large community, well documented, dev experience, current industry standard for modern php apps\cite{laravel-nova-docs}

\paragraph{Datenbank}
egal, hauptsache relational, standard und support durch laravel

\subsubsection{Frontend}

\paragraph{Chart.js}
gute doku, umfangreich aber nicht zu umfangreich
Vergleiche d3 und google charts

\paragraph{Polling}
Polling anstatt sockets, weil keine wirkliche Echtzeit notwendig, alle 5 sekunden genügt und erzeugt vermutlich weniger last auf dem server als

\subsubsection{Hosting}
Im Angebotsumfang sind zwei Arten eines Hostings enthalten, daher sind beide umgesetzt und vorbereitet.

\paragraph{Containerized}
Die besten Skalierungsmöglichkeiten bieten containerbasierte Anwendungen, es wird der Industriestandard Docker verwendet.
Auf Basis von Terraform, einem IaC (Infrastructure as Code) Tool, ist eine komplette Landschaft definiert.
Nach der Anbindung eines Kubernetes Clusters kann, mit einem CLI Befehl, das Hosting einer Test- und einer Produktivumgebung erstellt werden und bei Bedarf verändert werden.
Ebenfalls sind automatische Updates, durch das Bauen neuer Images und anschließende Rolling Updates, in der CI Pipeline bei GitLab implementiert.

\paragraph{vServer}
Ein klassischeres Hosting hat sich im Projektverlauf als unkomplizierter gezeigt.
Ein vServer ist durch einen Entwickler selbst schnell und übersichtlich verwaltbar.
Die Komplexität eines Clusters und der damit verbundene Overhead in der Administration entfällt.

Außerdem gibt es keinen Overhead durch Container, man erhält die volle Leistung der gebuchten Ressourcen, dies spart langfristig Kosten.

\subparagraph{Laravel Forge}
Durch den Einsatz von Laravel bietet sich die Verwendung von \enquote{Laravel Forge}\cite{laravel-forge}, einem First Party Server Management Tool, an.
Im aktuellen Projekt ergaben sich durch die Nutzung von Forge vor allem folgende Vorteile:
\begin{itemize}
    \item ein moderner und robuster Hosting Stack wird automatisiert verwaltet
    \item Push To Deploy ermöglicht schnelle Updates in Test- und Produktivsysteme
    \item automatisiert eingebundene TLS Zertifikate mittels LetsEncrypt
    \item Task Scheduling: Übersichtliche und schnelle Konfiguration durch Entwickler möglich
    \item Log File Access: gesammelter Zugriff auf alle relevanten Logfiles inklusive \enquote{Log Tailing}
\end{itemize}

Zukünftig können folgende Features von Vorteil sein:
\begin{itemize}
    \item Automatisierte Datenbankbackups
    \item Laravel Octane Support - neues Paradigma durch Stateful Server und dadurch schnellere Application Performance\cite{laravel-octane}
\end{itemize}

Als Hostinganbieter unterstützt Forge, DigitalOcean, Linode Cloud, AWS, Vultr und Hetzner Cloud.

\subparagraph{Hetzner Cloud}
Die Wahl auf die Hetzner Cloud als Hostingprovider fiel eindeutig durch mehrere Vorteile:
\begin{itemize}
    \item schnelles Up- und Downscaling möglich - von einer CPU mit 2 GB RAM hoch bis auf 16 CPUs und 32 GB RAM
    \item automatisierte Backups des ganzen Servers für kleinen Aufpreis möglich
    \item günstige Preise gegenüber den anderen Anbietern
    \item Datenschutz durch Rechenzentren in Deutschland
    \item örtliche Nähe des Rechenzentrums zu GRAW - beides in Nürnberg
\end{itemize}
