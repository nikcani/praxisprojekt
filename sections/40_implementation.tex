\section{Durchführung}


\subsection{Technische Umsetzung}

\subsubsection{Beschreibung der entwickelten Software}
TODO: schreiben (auf Vor- und Nachteile durch Nova eingehen)

\subsubsection{Probleme und Lösungen}
TODO: ausschreiben
- anscheinend verlassene Library
-- https://github.com/opis/json-schema
-- https://github.com/opis/json-schema/issues/123
- Styling von Custom Components: Learning - import der Nova Components mit vielen ../../ möglich, aber vermutlich nicht good practise, aber funktioniert gut, möglicherweise Probleme in der Zukunft, da nicht auf Nutzen als Component Lib ausgelegt und daher unerwartete Breaking changes
- infinite loading problem bei tabellen
- blick ins wiki ob da noch was steht!!!!


\subsection{Nutzwertanalyse Laravel Admin Panel}
TODO:
Schritt 1: Organisation des Arbeitsumfelds und Planung
- LESEN und DEFINIEREN

Schritt 2: Ziel und Entscheidungsproblem
- LESEN und DEFINIEREN

Schritt 3: Auswahl der Entscheidungsalternativen
- die Populärsten nach https://blog.forestadmin.com/the-guide-to-laravel-admin-panels/

Schritt 4: Bestimmung der Entscheidungskriterien
- https://blog.forestadmin.com/the-guide-to-laravel-admin-panels als Basis:
-- genug Ressourcen für Individualentwicklung
-- Konfiguration durch technisch weniger versierte Personen; also Low Code
-- Projektziel: MVP oder skalierbare Architektur
- aber erweitern durch eigene kriterien:
-- Lizenzmodell
--- Preis im MVP Projekt
--- Preis über Produktlaufzeit

Schritt 5: Gewichtung der Entscheidungskriterien
- LESEN und GEWICHTEN

Schritt 6: Skalen und Bewertungsvorschriften
- LESEN und VERSTEHEN

Schritt 7: Bewertung der Entscheidungskriterien
- Laravel Nova
-- https://nova.laravel.com
-- rapid dev time, good structure, extensible through components
- Filament
-- https://filamentphp.com/
- Backpack
-- https://backpackforlaravel.com/
- Voyager
-- https://voyager.devdojo.com/
- Quick Admin Panel
-- https://quickadminpanel.com/
-- Einmalige Code Generierung
-- alter Stack (jQuery); neuere Stacks (Vue.js oder Tailwind) noch Beta und wenige Features
-- viele Freiheiten nach erster Erstellung, dafür dann auch langsamer
- Forest Admin
-- https://www.forestadmin.com/
-- SAAS
-- Low Code (z.B. UI Layout Editor)

Schritt 8: Berechnung des Nutzwerts/Scores
- LESEN und BERECHNEN

Schritt 9: Sensitivitätsanalyse . . . . . . . . . . . . . . . . . . . . . . . . . . . . 81
- LESEN und NOTWENDIGKEIT klären
