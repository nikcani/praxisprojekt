\section{Durchführung}

\subsection{Technische Umsetzung}

\subsubsection{Beschreibung der entwickelten Software}
TODO: schreiben (auf Vor- und Nachteile durch Nova eingehen)

\subsubsection{Probleme und Lösungen}
anscheinend verlassene Library mit Fehlern: https://github.com/opis/json-schema/issues/123

Styling von Custom Components: Learning - import der Nova Components mit vielen ../../ möglich, aber vermutlich nicht good practise, aber funktioniert gut, möglicherweise Probleme in der Zukunft, da nicht auf Nutzen als Component Lib ausgelegt und daher unerwartete Breaking changes

infinite loading problem bei tabellen

blick ins wiki ob da noch was steht!

\subsection{Nutzwertanalyse Laravel Admin Panel}

\subsubsection{Organisation des Arbeitsumfelds und Planung}
LESEN und DEFINIEREN

\subsubsection{Ziel und Entscheidungsproblem}
LESEN und DEFINIEREN

\subsubsection{Auswahl der Entscheidungsalternativen}
Die populärsten Lösungen werden auf Basis eines Artikels ausgewählt, in dem sich eine Alternative selbst mit Anderen vergleich (Siehe https://blog.forestadmin.com/the-guide-to-laravel-admin-panels).
Es ist davon auszugehen, dass ein Anbieter in diesem Bereich die eigene Konkurrenz gut kennt.

\subsubsection{Bestimmung der Entscheidungskriterien}
BASIS: https://blog.forestadmin.com/the-guide-to-laravel-admin-panels
genug Ressourcen für Individualentwicklung
Konfiguration durch technisch weniger versierte Personen; also Low Code
Projektziel: MVP oder skalierbare Architektur

ERWEITERN durch eigene kriterien:
Lizenzmodell
Preis im MVP Projekt
Preis über Produktlaufzeit

\subsubsection{Gewichtung der Entscheidungskriterien}
LESEN und GEWICHTEN

\subsubsection{Skalen und Bewertungsvorschriften}
LESEN und VERSTEHEN

\subsubsection{Bewertung der Entscheidungskriterien}
Laravel Nova
https://nova.laravel.com
rapid dev time, good structure, extensible through components

Filament
https://filamentphp.com/

Backpack
https://backpackforlaravel.com/

Voyager
https://voyager.devdojo.com/

Quick Admin Panel
https://quickadminpanel.com/

Einmalige Code Generierung
alter Stack (jQuery); neuere Stacks (Vue.js oder Tailwind) noch Beta und wenige Features
viele Freiheiten nach erster Erstellung, dafür dann auch langsamer

Forest Admin
https://www.forestadmin.com/
SAAS
Low Code (z.B. UI Layout Editor)

\subsubsection{Berechnung des Nutzwerts/Scores}
LESEN und BERECHNEN

\subsubsection{Sensitivitätsanalyse}
LESEN und NOTWENDIGKEIT klären
