\newpage

\subsection{Nutzwertanalyse einiger Laravel Admin-Panels}

\subsubsection{Entscheidungsproblem}
Welches Laravel Admin-Panel ist für das konkrete Projekt der Sounding Console am geeignetsten?

\subsubsection{Entscheidungsalternativen}
Die populärsten Lösungen werden auf Basis eines Artikels ausgewählt, in dem sich eine Alternative selbst mit Anderen vergleich (Siehe~\cite{the-guide-to-laravel-admin-panels}).
Es ist davon auszugehen, dass ein Anbieter in diesem Bereich, die eigene Konkurrenz gut kennt.

\subsubsection{Bestimmung der Entscheidungskriterien}
Aus technischer Sicht ist die Anpassbarkeit der Software und die technische Flexibilität des gesamten Stacks.
Außerdem ist ein flexibles Hosting eine klare Projektanforderung und muss daher beachtet werden.
Es ergeben sich daher die Kriterien \enquote{Anpassbarkeit}, \enquote {Flexibilität Tech Stack} und \enquote{Hostingflexibilität}.

Grundsätzlich sind, im betrieblichen Umfeld, die Entwicklungskosten primär ausschlaggebend für neue Projekte.
Zusätzlich sind die laufenden Kosten, für die Wirtschaftlichkeit eines Produktes, entscheidend.
Folgende Kriterien wurden aus betrieblicher Sicht herausgearbeitet: \enquote{Time To Production}, \enquote{Eignung Lizenzmodell}, \enquote{Lizenzkosten (MVP)} und \enquote{Lizenzkosten (Produktlebenszyklus)}.

\subsubsection{Gewichtung der Entscheidungskriterien}
\begin{table}[h]
    \begin{center}
        \begin{tabular}{|c|l|l|}
            \hline
            A & 20\% & Anpassbarkeit                      \\ \hline
            B & 10\% & Flexibilität Tech Stack            \\ \hline
            C & 20\% & Hostingflexibilität                \\ \hline
            D & 20\% & Time To Production                 \\ \hline
            E & 10\% & Eignung Lizenzmodell               \\ \hline
            F & 10\% & Lizenzkosten (MVP)                 \\ \hline
            G & 10\% & Lizenzkosten (Produktlebenszyklus) \\ \hline
        \end{tabular}
    \end{center}
    \caption{Gewichtung der Entscheidungskriterien}
    \label{tab:gewichtung-der-entscheidungskriterien}
\end{table}

\newpage

\subsubsection{Bewertung der Entscheidungskriterien}
Die Bewertung basiert auf Schätzungen eines Entwicklers nach Recherche der verglichenen Tools.
In einem größeren Projekt sollten mehrere Personen Schätzungen durchführen, um die Qualität des Ergebnisses zu erhöhen.

Es wird eine 10er-Skala bei der Bewertung je Kriterium angewendet, hohe Werte sind positiv definiert.
Die Kriterien sind aus Platzgründen in der Tabelle nur in gekürzter Form benannt.

\begin{table}[h]
    \resizebox{\textwidth}{!}{
        \begin{tabular}{|c|l|c|c|c|c|c|c|c|}
            \hline
            &
            & \textbf{\href{https://nova.laravel.com}{Laravel Nova}}
            & \textbf{\href{https://filamentphp.com}{filament}}
            & \textbf{\href{https://backpackforlaravel.com}{Backpack}}
            & \textbf{\href{https://voyager.devdojo.com}{Voyager}}
            & \textbf{\href{https://quickadminpanel.com}{Quick Admin Panel}}
            & \textbf{\href{https://forestadmin.com}{Forest Admin}}
            \\ \hline
            A & \textbf{Anpassbarkeit}         & 6  & 8  & 9  & 8  & 10 & 3 \\ \hline
            B & \textbf{Tech Stack}            & 6  & 7  & 7  & 7  & 7  & 2 \\ \hline
            C & \textbf{Hosting}               & 10 & 10 & 10 & 10 & 10 & 5 \\ \hline
            D & \textbf{Time}                  & 9  & 9  & 9  & 8  & 5  & 8 \\ \hline
            E & \textbf{Lizenzmodell}          & 10 & 10 & 10 & 10 & 10 & 1 \\ \hline
            F & \textbf{€ MVP}                 & 8  & 10 & 8  & 10 & 9  & 5 \\ \hline
            G & \textbf{€ Produktlebenszyklus} & 10 & 10 & 10 & 10 & 10 & 3 \\ \hline
        \end{tabular}
    }
    \caption{Bewertung der Entscheidungskriterien}
    \label{tab:bewertung-der-entscheidungskriterien}
\end{table}

\subsubsection{Nutzwertberechnung}
B = Bewertung | G = Gewichtung | GB = Gewichtete Bewertung\\
QAP = Quick Admin Panel

\begin{table}[h]
    \resizebox{\textwidth}{!}{
        \begin{tabular}{|c|r|c|c|r|c|c|r|c|c|r|c|c|r|c|c|r|c|c|}
            \hline
            & \multicolumn{3}{|c|}{\textbf{\href{https://nova.laravel.com}{Laravel Nova}}}
            & \multicolumn{3}{|c|}{\textbf{\href{https://filamentphp.com}{filament}}}
            & \multicolumn{3}{|c|}{\textbf{\href{https://backpackforlaravel.com}{Backpack}}}
            & \multicolumn{3}{|c|}{\textbf{\href{https://voyager.devdojo.com}{Voyager}}}
            & \multicolumn{3}{|c|}{\textbf{\href{https://quickadminpanel.com}{QAP}}}
            & \multicolumn{3}{|c|}{\textbf{\href{https://forestadmin.com}{Forest}}}
            \\ \hline
            - & B  & G    & GB  & B  & G    & GB  & B  & G    & GB  & B  & G    & GB  & B  & G    & GB  & B & G    & GB  \\ \hline\hline
            A & 6  & 20\% & 1,2 & 8  & 20\% & 1,6 & 9  & 20\% & 1,8 & 8  & 20\% & 1,6 & 10 & 20\% & 2,0 & 3 & 20\% & 0,6 \\ \hline
            B & 6  & 10\% & 0,6 & 7  & 10\% & 0,7 & 7  & 10\% & 0,7 & 7  & 10\% & 0,7 & 7  & 10\% & 0,7 & 2 & 10\% & 0,2 \\ \hline
            C & 10 & 20\% & 2,0 & 10 & 20\% & 2,0 & 10 & 20\% & 2,0 & 10 & 20\% & 2,0 & 10 & 20\% & 2,0 & 5 & 20\% & 1,0 \\ \hline
            D & 9  & 20\% & 0,9 & 9  & 20\% & 1,8 & 9  & 20\% & 1,8 & 8  & 20\% & 1,6 & 5  & 20\% & 1,0 & 8 & 20\% & 1,6 \\ \hline
            E & 10 & 10\% & 1,0 & 10 & 10\% & 1,0 & 10 & 10\% & 1,0 & 10 & 10\% & 1,0 & 10 & 10\% & 1,0 & 1 & 10\% & 0,1 \\ \hline
            F & 8  & 10\% & 0,8 & 10 & 10\% & 1,0 & 8  & 10\% & 0,8 & 10 & 10\% & 1,0 & 9  & 10\% & 0,9 & 5 & 10\% & 0,5 \\ \hline
            G & 10 & 10\% & 1,0 & 10 & 10\% & 1,0 & 10 & 10\% & 1,0 & 10 & 10\% & 1,0 & 10 & 10\% & 1,0 & 3 & 10\% & 0,3 \\ \hline
            & \multicolumn{3}{|r|}{\textbf{7,5}}
            & \multicolumn{3}{|r|}{\textbf{9,1}}
            & \multicolumn{3}{|r|}{\textbf{9,1}}
            & \multicolumn{3}{|r|}{\textbf{8,9}}
            & \multicolumn{3}{|r|}{\textbf{8,6}}
            & \multicolumn{3}{|r|}{\textbf{4,3}}
            \\ \hline
        \end{tabular}
    }
    \caption{Nutzwertberechnung}
    \label{tab:nutzwertberechnung}
\end{table}
