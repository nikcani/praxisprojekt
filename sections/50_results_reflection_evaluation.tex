\section{Reflektion}
Die Kommunikation bezüglich der Anforderungen zwischen Auftraggeber und Entwickler verlief schnell und zielorientiert.
Vor allem das daraus schnell verfügbare verschriftliche Leistungspaket ermöglichte eine stringente Implementierung.

TODO: Entgegen ersten Plänen: sondehub-tracker kann zu viel nicht benötigtes, lieber direkt mit leaflet
TODO: Entgegen ersten Plänen: thermodynamische Diagramme, insbedondere Skew-T sollten aus sondehub-tracker genutzt werden, eine einzelne js lib wurde gefunden und eingebunden, durch reverse engeneering der demo. Eine doku gibt es nicht.
TODO: Entgegen ersten Plänen: classical hosting weil gut verständlich, schnell durch forge, stabil und günstig, außerdem sogar recht gut skalierbar durch hetzner cloud

Die Kommunikation bezüglich der Schnittstellen zu GRAWMET verlief inhaltlich gut, durch sehr begrenzte Entwicklungsressourcen beim Auftraggeber allerdings eher langsam.
Daher ist bisher nur die Anbindung für laufende Flüge, jedoch nicht der Import alter Flüge.
Dieser wurde zeitlich nicht priorisiert und bleibt daher, aufgrund fehlender Informationen zu bestehenden Datenformaten, vorerst offen.

Trotz guten und sehr konkreten Absprachen bezüglich der Bodenstationen, stellte sich erst im späteren Projektverlauf heraus, dass es auch bewegliche Bodenstationen gibt.
Durch die Abwesenheit von Produktivdaten konnte jedoch schnell das Schema angepasst werden und eine 1:n-Beziehung, zu einer neuen Positionsressource, implementiert werden.
