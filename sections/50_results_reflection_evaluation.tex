\section{Reflektion}

\subsection{Kommunikation}
Die Kommunikation bezüglich der Anforderungen zwischen Auftraggeber und Entwickler verlief schnell und zielorientiert.
Vor allem das daraus schnell verfügbare verschriftliche Leistungspaket ermöglichte eine stringente Implementierung.

Die Kommunikation bezüglich der Schnittstellen zu GRAWMET verlief inhaltlich gut, durch sehr begrenzte Entwicklungsressourcen beim Auftraggeber allerdings eher langsam.
Daher ist bisher nur die Anbindung für laufende Flüge, jedoch nicht der Import alter Flüge.
Dieser wurde zeitlich nicht priorisiert und bleibt daher, aufgrund fehlender Informationen zu bestehenden Datenformaten, vorerst offen.

Trotz guten und sehr konkreten Absprachen bezüglich der Bodenstationen, stellte sich erst im späteren Projektverlauf heraus, dass es auch bewegliche Bodenstationen gibt.
Durch die Abwesenheit von Produktivdaten konnte jedoch schnell das Schema angepasst werden und eine 1:n-Beziehung, zu einer neuen Positionsressource, implementiert werden.

\subsection{Zielerreichung}
Das grundsätzliche Projektziel wurde vollständig erreicht.
Die entwickelte Sounding Console ist voll funktionsfähig und verhält sich, in der generellen Benutzererfahrung, vergleichbar zu ausgereiften Applikationen.
Das Interface reagiert schnell auf Benutzereingaben und ist grundsätzlich verständlich.

Der geplante Funktionsumfang wurde, bis auf wenige Ausnahmen, vollständig abgedeckt.
Teilweise wurden sogar Features implementiert, die im Funktionsumfang nicht, oder nur optional, vorgesehen waren.
