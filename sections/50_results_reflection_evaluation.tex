\section{Reflektion}

\subsection{Kommunikation}
Die Kommunikation bezüglich der Anforderungen zwischen Auftraggeber und Entwickler verlief schnell und zielorientiert.
Das anhand dieser Absprachen entwickelte schriftliche Leistungspaket ermöglichte eine stringente Implementierung.

Die Kommunikation bezüglich der Schnittstellen zu GRAWMET verlief inhaltlich gut.
Aufgrund sehr begrenzter Entwicklungsressourcen beim Auftraggeber auf der Zeitschiene jedoch recht langsam.
Insofern ist bisher nur die Anbindung für laufende Flüge implementiert, jedoch nicht der Import alter Flüge.
Dieser wurde zeitlich nicht priorisiert und bleibt auch aufgrund fehlender Informationen zu bestehenden Datenformaten vorerst offen.

Trotz guter und sehr konkreter Absprachen bezüglich der Bodenstationen, stellte sich im späteren Projektverlauf heraus, dass es auch bewegliche Bodenstationen gibt.
Durch die Abwesenheit von Produktivdaten in der Datenbank konnte jedoch schnell das Schema angepasst werden und eine 1:n-Beziehung zu einer neuen Positionsressource implementiert werden.

\subsection{Zielerreichung}
Das Projektziel wurde vollständig erreicht.
Die entwickelte Sounding Console ist voll funktionsfähig und verhält sich, in der generellen Benutzererfahrung, vergleichbar zu ausgereiften Applikationen.
Das Interface reagiert schnell auf Benutzereingaben und ist grundsätzlich verständlich.

Der geplante Funktionsumfang wurde, bis auf wenige Ausnahmen, vollständig abgedeckt.
Teilweise konnten zudem Features implementiert werden, die im Funktionsumfang nicht, oder nur optional, vorgesehen waren.
